%%%%%%%%%%%%%%%%%%%%%%%%%%%%%%%%%%%%%%%%%%%%%%%%%%%%%%%%%%%%%%%%%%%%%
%%                                                                 %%
%% Please do not use \input{...} to include other tex files.       %%
%% Submit your LaTeX manuscript as one .tex document.              %%
%%                                                                 %%
%% All additional figures and files should be attached             %%
%% separately and not embedded in the \TeX\ document itself.       %%
%%                                                                 %%
%%%%%%%%%%%%%%%%%%%%%%%%%%%%%%%%%%%%%%%%%%%%%%%%%%%%%%%%%%%%%%%%%%%%%

%%\documentclass[referee,sn-basic]{sn-jnl}% referee option is meant for double line spacing

%%=======================================================%%
%% to print line numbers in the margin use lineno option %%
%%=======================================================%%

%%\documentclass[lineno,sn-basic]{sn-jnl}% Basic Springer Nature Reference Style/Chemistry Reference Style

%%======================================================%%
%% to compile with pdflatex/xelatex use pdflatex option %%
%%======================================================%%

%%\documentclass[pdflatex,sn-basic]{sn-jnl}% Basic Springer Nature Reference Style/Chemistry Reference Style

%%\documentclass[sn-basic]{sn-jnl}% Basic Springer Nature Reference Style/Chemistry Reference Style
\documentclass[sn-mathphys]{sn-jnl}% Math and Physical Sciences Reference Style
%%\documentclass[sn-aps]{sn-jnl}% American Physical Society (APS) Reference Style
%%\documentclass[sn-vancouver]{sn-jnl}% Vancouver Reference Style
%%\documentclass[sn-apa]{sn-jnl}% APA Reference Style
%%\documentclass[sn-chicago]{sn-jnl}% Chicago-based Humanities Reference Style
%%\documentclass[sn-standardnature]{sn-jnl}% Standard Nature Portfolio Reference Style
%%\documentclass[default]{sn-jnl}% Default
%%\documentclass[default,iicol]{sn-jnl}% Default with double column layout

%%%% Standard Packages
%%<additional latex packages if required can be included here>
    \usepackage{etoolbox}
    \makeatletter
\patchcmd{\ps@headings}
{\hbox to \hsize{\hfill Springer Nature 2021 \LaTeX\ template\hfill}}
{\hbox to \hsize{}}
{}
{}
\patchcmd{\ps@headings}
{\hbox to \hsize{\hfill Springer Nature 2021 \LaTeX\ template\hfill}}
{\hbox to \hsize{}}
{}
{}
\patchcmd{\ps@titlepage}
{\hbox to \hsize{\hfill Springer Nature 2021 \LaTeX\ template\hfill}}
{\hbox to \hsize{}}
{}
{}
    \makeatother
    
\usepackage{ltablex}
\usepackage{multirow}
\usepackage{dcolumn}
\usepackage{lscape}
\usepackage{longtable}
\usepackage{color, colortbl}
\usepackage{array, booktabs, ragged2e}
\usepackage{makecell}

\usepackage{textpos}

\definecolor{lightgray}{rgb}{0.83, 0.83, 0.83}
%%%%

%%%%%=============================================================================%%%%
%%%%  Remarks: This template is provided to aid authors with the preparation
%%%%  of original research articles intended for submission to journals published 
%%%%  by Springer Nature. The guidance has been prepared in partnership with 
%%%%  production teams to conform to Springer Nature technical requirements. 
%%%%  Editorial and presentation requirements differ among journal portfolios and 
%%%%  research disciplines. You may find sections in this template are irrelevant 
%%%%  to your work and are empowered to omit any such section if allowed by the 
%%%%  journal you intend to submit to. The submission guidelines and policies 
%%%%  of the journal take precedence. A detailed User Manual is available in the 
%%%%  template package for technical guidance.
%%%%%=============================================================================%%%%

\jyear{2023}%

%% as per the requirement new theorem styles can be included as shown below
\theoremstyle{thmstyleone}%
\newtheorem{theorem}{Theorem}%  meant for continuous numbers
%%\newtheorem{theorem}{Theorem}[section]% meant for sectionwise numbers
%% optional argument [theorem] produces theorem numbering sequence instead of independent numbers for Proposition
\newtheorem{proposition}[theorem]{Proposition}% 
%%\newtheorem{proposition}{Proposition}% to get separate numbers for theorem and proposition etc.

\theoremstyle{thmstyletwo}%
\newtheorem{example}{Example}%
\newtheorem{remark}{Remark}%

\theoremstyle{thmstylethree}%
\newtheorem{definition}{Definition}%

\raggedbottom
%%\unnumbered% uncomment this for unnumbered level heads
\usepackage{fancyhdr}
\pagestyle{fancy}
\fancyhead{}
%\renewcommand{\headrulewidth}{0pt}


\begin{document}



\title[Article Title]{Cattle, Cadaster, \& Conflict: Settlement Growth and Social Conflict in Early Colonial New England 1620-1680}

%%=============================================================%%
%% Prefix	-> \pfx{Dr}
%% GivenName	-> \fnm{Joergen W.}
%% Particle	-> \spfx{van der} -> surname prefix
%% FamilyName	-> \sur{Ploeg}
%% Suffix	-> \sfx{IV}
%% NatureName	-> \tanm{Poet Laureate} -> Title after name
%% Degrees	-> \dgr{MSc, PhD}
%% \author*[1,2]{\pfx{Dr} \fnm{Joergen W.} \spfx{van der} \sur{Ploeg} \sfx{IV} \tanm{Poet Laureate} 
%%                 \dgr{MSc, PhD}}\email{iauthor@gmail.com}
%%=============================================================%%

\author{{\centering \fnm{Eric H} \sur{Wilhelm} \\ \href{ewilhelm@gmu.edu}{ewilhelm@gmu.edu} \\ George Mason University}}

%\affil{ George Mason University}



%%==================================%%
%% sample for unstructured abstract %%
%%==================================%%

\abstract{Property rights are secure, and violence over land can be attenuated when the treatment and delineation of the property are consistent, stable, and salient to each party. Land-use stability becomes strained as the area of contested land between two rival parties expands - when one party (or group) is perceived as asymmetrically and rapidly accumulating land at another's expense. While relations between Algonquian tribes and English settlers were generally peaceful in the first half of the 17th century, subsequent colonial growth accelerated in the late 17th century and lead to violent conflict. This period of violence culminated in King Philip's War of 1675-1676, the most devastating conflict of early colonial American development. Using probate data covering 56 settlements in New England to measure the growth of farmers as a proxy for colonial territorial growth, I find English settlements that doubled in size were 8\% more likely to be damaged or destroyed by an Algonquian tribe. The correlation between settlement growth and conflict also holds after controlling for initial settlement size.}

%%================================%%
%% Sample for structured abstract %%
%%================================%%

% \abstract{\textbf{Purpose:} The abstract serves both as a general introduction to the topic and as a brief, non-technical summary of the main results and their implications. The abstract must not include subheadings (unless expressly permitted in the journal's Instructions to Authors), equations or citations. As a guide the abstract should not exceed 200 words. Most journals do not set a hard limit however authors are advised to check the author instructions for the journal they are submitting to.
% 
% \textbf{Methods:} The abstract serves both as a general introduction to the topic and as a brief, non-technical summary of the main results and their implications. The abstract must not include subheadings (unless expressly permitted in the journal's Instructions to Authors), equations or citations. As a guide the abstract should not exceed 200 words. Most journals do not set a hard limit however authors are advised to check the author instructions for the journal they are submitting to.
% 
% \textbf{Results:} The abstract serves both as a general introduction to the topic and as a brief, non-technical summary of the main results and their implications. The abstract must not include subheadings (unless expressly permitted in the journal's Instructions to Authors), equations or citations. As a guide the abstract should not exceed 200 words. Most journals do not set a hard limit however authors are advised to check the author instructions for the journal they are submitting to.
% 
% \textbf{Conclusion:} The abstract serves both as a general introduction to the topic and as a brief, non-technical summary of the main results and their implications. The abstract must not include subheadings (unless expressly permitted in the journal's Instructions to Authors), equations or citations. As a guide the abstract should not exceed 200 words. Most journals do not set a hard limit however authors are advised to check the author instructions for the journal they are submitting to.}

\keywords{Political economy; Institutions; Property rights; Social conflict; Colonialism}

\pacs[JEL Classification]{D23, D74, N41, O1, Q34, R14}

%%\pacs[MSC Classification]{35A01, 65L10, 65L12, 65L20, 65L70}

\maketitle

\newpage

\section{Introduction}\label{sec1}

Social norms and formal rules help define property rights and resolve property conflicts over land use. Historically, these institutions have either been relegated to formal states %\todo{Any other \#statecapacity literature worth mentioning?}  
(Acemoglu \& Robinson, 2019; Johnson \& Koyama, 2017; and Dinecco \& Katz, 2016) or deferred to stateless channels of resolution (Candela \& Geloso, 2020). The violence-reducing characteristics of these institutions include: 1) defining property rights over space and time, 2) adjudicating contests over property, and 3) enforcing those resolutions after adjudication. Areas with little to no formal state institutions, such as the early colonial period in North America between European settlers and Algonquian Indians, generally relied on the first characteristic and had few viable alternatives for the latter two. 

During the period of rapid colonial expansion in the mid-17th century, the mutual understanding of land use between English settlers and Algonquian tribes changed. Rapid colonial population growth and expansion after the arrival of the {\em Mayflower} in 1620 made the process of defining and redefining property rights costly through peaceful means. The sharp change in relative growth (between the English and Algonquians) expanded the area of contested land. The area of contested land grew in proportion to the expansion of English settlements.

The type of institutions that addressed and resolved potential land conflict impacted the likelihood and severity of violence in colonial North America. In contrast to the English colonial experience in New England, French colonists saw little to no violent conflict with Acadians and Mikmaqs. The rules of collective decision-making for settling land disputes favored consensus and greatly reduced the returns to conflict (Candela \& Geloso, 2020). All parties had to come to a collective agreement, eschewing the formation of special interests who could potentially benefit from fighting and spill the external costs of collateral damage onto the rest of the population. 

Competing interests over natural resources (like beaver) constrained the types of institutions that emerged in Canadian North America. European settlements around Hudson Bay had competing interests and varying property rights institutions. Native Americans near Fort Albany and the York Factory faced competition in the beaver fur trade from the French (Carlos \& Lewis 1993, 1999, 2001). Hudson Bay Company managers increased the price of furs in those areas which led to more Native Americans entering the market and deplete the beaver population more quickly. Fort Churchill did not face the same level of competition in the supply of beaver pelt. Prices for furs in that region were more stable, and the beaver population was not depleted as quickly (ibid). Both colonial episodes in French Acadia and the Hudson Bay Company demonstrate how initial settlements, endowments, and incentives impacted the evolution of institutions that helped define property rights and resolve property conflicts. The types of institutions that evolved had varying success at mitigating violent conflict over land and managing natural resources. 

This was not the case in early colonial New England which did not have a similar type of consensus-building institution to effectively resolve conflicts over land use. Early English colonial development was characterized by an inchoate delineation of boundaries and rapid colonist population growth at varying proximities to Algonquian settlements. And English settlements that grew more rapidly in closer proximity to Algonquian settlements were more likely to be targeted in conflict.

Recent literature has looked at the relationship between land conflict and contract choice in a modern context. %\todo{Let me know if you think this paragraph about modern-day impact is relevant/useful. I can drop it. I also don't know where to put it.}. 
Lee Alston and Bernardo Mueller (2010) analyze how land conflict and initial property endowments impact subsequent contract choice and the type of tenancy arrangement chosen in Brazil. Similarly, Conning and Robinson (2007) examine how property insecurity impacts the type of agricultural organization selected among competing claimants. They use a model of potential land reform to demonstrate how an agent's expectations of property insecurity, instigated by the likelihood of land reform, are likely to modify their current choice of contract. Both papers measure the impact of property insecurity on subsequent contract choice. This paper examines the relationship in reverse. How does property insecurity combined with an increase in the area of contested land lead to conflict?

\section{Historical Background}\label{sec2}

After Jamestown was founded in 1607, English emigrants settled in the colonies of Plymouth and Massachusetts Bay. Pilgrims and Puritan settlers left Holland and England to escape religious-conformity restrictions imposed by the state religion of their home country. Disheartened by religious persecution, they wished to preserve their English identity and brought with them their own form of self-governance and legal framework (Winslow, 1646). The attitude of European colonizers towards {\em terra nullius} (``land with no owners") was varied (Pagden, 2015). 
%The General Court of Massachusetts decreed in 1648 ``that anyone who received a grant of land by what the court termed {\em vacuum domicilium} but did not build on or `improve' it within a space of three years would lose it" (ibid.). 
English setters' legal justification for a formal expression of property rights was established through agricultural development. The land had to be tilled, sown, harvested, or grazed. Those acts of land improvement, from the English perspective, laid their claim to ownership.

The area of land spanning the coastline of Maine to Long Island Sound included many Algonquian peoples from the Massachusett, settled around Massachusetts Bay, to the Wampanoag, settled around Cape Cod Bay, and the Narragansett in the west of modern day Rhode Island (Washburn, 1989). Over the course of the 17th century, English settlers established various diplomatic, commercial, and religious connections with surrounding tribes, sachems (male leaders), and sunksquaws (female leaders with great authority). Settlers and Natives conversed, intermarried, and formed treaties (Schultz, 1999 and Warren, 2018). 

They also exchanged goods and land. As the children and grand-children of England's first ``Great Migration'' came of age, by 1670 the English population burgeoned to over 60,000 people, almost double the Native New England population (Silverman, 2019). Early interactions between Algonquians and the English (1620-1670) were pacific\footnote{Descriptions of different types of European-Algonquian interactions are provided in Appendix A.}. What sparked tensions and violent conflict was the rapid rise of English pastures and farms for husbandry - a dramatic increase in the demand for land. 

At first, English settlers attempted to purchase land through various exchanges like manufactured goods or wampum, a string of beads used by Algonquian tribes as a form of currency (Brooks, 2018 and Schultz, 1999). These trades often took a strategic tone in the greater context of internecine power relations among European powers and rival Algonquian tribes (ibid). The source of conflict resulted from poorly defined property rights - the perceived encroachment or illegitimate purchase of land between English settlers and the Wampanoag, Narragansett, and Nipmuck. More land was contested.

From the perspective of English settlers, an open expanse of land, a significantly depopulated Algonquian territory devastated by disease prior to the arrival of the English (Steckel et. al., 2002), and initial comity with the Wampanoag made the subject of contract choice relatively simple. Under consent and charter from the Crown, most English requests for land were made through purchase (Pagden, 2015 and Roback, 1992). Challenges over occupancy with surrounding Algonquian tribes were generally met through treaty, gift, or exchange\footnote{Since their arrival, English settlers treated native land as {\em de facto} sovereign territory of American Indians. It would not be until the Treaty of Paris when the Royal Proclamation of 1763 formally ceded {\em de iure} autonomy to ``several Nations or Tribes of Indians".}. However, poorly defined property rights, ambiguous land-use arrangements, and rapid colonial settlement growth through land-intensive animal husbandry set the stage for one of the most devastating periods of conflict between the English and Algonquians in the Americas.

The Wampanoag, Narragansett, and Nipmuck settled in-land, away from the coastlines, and along well-protected rivers and marshes (Washburn, 1989). Algonquian tribes had a deep knowledge of the terrain and exposed points of attack along coastlines where English colonists predominately settled. Generally, Algonquian forces were mobile and used their knowledge of the land to their advantage while the English typically fought in fixed points of defense near their settlements (Schultz, 1999). 

The introduction of livestock, expansion of land-intensive agriculture, and rapid settlement growth characterized early colonial development in New England. Four years after the arrival of the {\em Mayflower}, Edward Winslow, one of the early Pilgrim Fathers, brought from England ``three heifers and a bull, the first of any cattle of that kind in the land'' (Anderson, 1994, p. 602). The task of improving the land was met as a general measurement of their prosperity. By 1627, Plymouth Colony had accumulated - either through husbandry or import - up to ``fifteen animals, whose muscle power increased agricultural productivity'' (Anderson, 1979 and PARP). In order to accommodate this land-intensive form of agriculture, Plymouth began expanding its borders beyond the boundaries of the compact village established at the time of the first Thanksgiving. They traded manufactures and agricultural products with Massasoit, sachem of the Wampanoag and father of Metacomet, in exchange for more land.

The most lethal and politically potent of these manufactured goods were matchlock and flintlock firearms. The proliferation of guns was initially spurred by early colonizers (Silverman, 2016). Colonial governments were late to enforce the arms trade; and even when they began imposing restrictions on the movement of guns, black markets rose. The arms build up - predominantly spurred by Dutch, French, and English arms traders - revolutionized warfare for the Wampanoag. Metacomet, who the English called King Philip, saw the colonies as attempting to drive them off their land. Exacerbating tribal agitation, colonial governments banned the arms trade entirely and began confiscating arms from the Wampanoag (Lepore, 1998 and Silverman, 2016). The confiscation of arms and the perception of conquest came at a time of increased tensions between English colonists and the Wampanoag, Nipmuck, and Narragansett\footnote{The Narragansett remained neutral at the outset of King Philip's War and entered the conflict after peace negotiations with New England militias broke down.}. 

John Sassamon, a Massachusett raised by an English family, was a mediator with significant influence with both the English and Wampanoag. He served as an advisor to Metacomet and was in a good position to ease the rising tensions. It was not until Sassamon was found dead in Assawompset Swamp in the winter of 1676 when conflict erupted. While the nature of Sassamon's death has never been confirmed, three of Metacomet's closest advisors were arrested, tried, and executed by Plymouth colonists for Sassamon's murder later that summer (ibid). Metacomet understood the colonists' summary judgement and execution as a threat. It meant the Wampanoag were a subject people and beholden to a foreign form of justice (Silverman, 2016). King Philip's War had begun.

\section{Theoretical Model of Land-Use Conflict}\label{sec3}

%\subsection{This is an example for second level head---subsection head}\label{subsec2}

%\subsubsection{This is an example for third level head---subsubsection head}\label{subsubsec2}

Disagreements over land use can be resolved by jointly defining property rights in which both parties come to a resolution or lead to violent conflict. In a setting with unclear property rights and no mutually-agreed authority (or set of rules) to resolve disagreements over land use, the likelihood of violent conflict increases as the area of contestation - an overlapping geographic area (of uniform natural resource and strategic value) where both parties lay claim to the same plot of land - spreads. Absent formal institutions to resolve land-use disagreements, increases in the area of contestation between Algonquian tribal settlements and colonial English settlements are more likely to be targets of violent conflict.

Potential determinants of overlapping land claims during the period of colonial settlement in 17th century New England include 1) the initial settlement size of colonial towns and villages, 2) the distance between a colonial settlement and a rival, tribal village, and 3) colonial settlement growth. Generally, the area of potentially contested land increases with respect to the initial colonial settlement size, proximity to nearest tribal village, and colonial settlement growth. All else constant, settlement growth increases the area of contested land and the likelihood of conflict such that: 
\begin{equation}
logit(p_{conflict})= \alpha + \beta_{1}g + \epsilon_{1} 
\end{equation}
where $p_{conflict}$ is a binary variable indicating conflict (1,0). $g$ is settlement growth up until the period of conflict. The functional form is a generalized linear model in which conflict is expressed in terms of a probability of conflict:
$$
p_{conflict}=\frac{1}{1+e^{-\alpha-\beta_{1}g}}
$$

According to this functional form, the growth of colonial settlements increases the potential area of contestation between rivals and increases the likelihood of conflict. Other functional forms incorporating initial settlement size and proximity to tribal villages are also considered.
\begin{equation}
logit(p_{conflict})= \alpha + \beta_{1}g + \beta_{2}\chi + \epsilon
\end{equation}
\begin{equation}
logit(p_{conflict})= \alpha + \beta_{1}g + \beta_{2}\chi + \beta_{3}\psi + \epsilon
\end{equation}
where $\chi$ is the initial settlement size and $\psi$ is the distance to the nearest tribal settlement. 

Initial settlement size and subsequent growth may also have interaction effects with respect to the likelihood of conflict. The attacker may deem the defender too large to attack because of an overwhelming imbalance in the size of forces. Subsequent settlement growth rates also rely on a settlement's baseline size. Smaller settlements can grow more quickly compared to settlements that are founded with larger initial populations\footnote{This also assumes a wide variation in initial settlement sizes across colonial New England settlements.}. In order to account for the interaction effects between initial settlement size and subsequent settlement growth, the last functional form weighs each settlement's growth rate by initial settlement size.
\begin{equation}
logit(p_{conflict})= \alpha + \beta_{interaction}\frac{g}{\chi} + \epsilon 
\end{equation}
This last model attempts to measure the relationship between settlement growth and the probability of conflict given a settlement's initial size upon founding.

\section{Cross-Township Analysis}\label{sec4}

The models above describe how the geographic distribution and growth of English townships\footnote{I use ``township'' interchangeably with colonial ``settlement''. They both refer to colonial compact villages of New England. Probate records are delineated by township.} (defined by a change in settler population, $g$) increase the area of contested land, break down the possibility of peaceful resolution, and spur violent conflict.
%One of the motivating factors for conflict was the perception of rapid settlement expansion. 
In order to measure the determinants of land-use conflict, I examine the relationship between the rate of colonial expansion and the likelihood of conflict at the township level. I first quantify the magnitude of early colonial expansion across all settlements in colonial New England and identify whether a township was damaged or destroyed by rival tribes from 1620-1676. Estimates of the potential determinants of the likelihood of conflict\footnote{Conflict is defined as whether a colonial settlement was damaged or destroyed.} are projected using a generalized linear logistic model.

A first-best measurement of colonial expansion is the change in land area for each township over the preceding decades. Growth in colonial territory expanded the area of potentially contested land and spurred more land-use disagreements between the English and Algonquians. Unfortunately, data on cadastral \footnote{The formal surveying and public assignment of cadasters do not begin historically until later in the 17th century. A ``cadaster" in this context is simply a publicly salient delineation of property.} size and development for each township is not readily available at the cross-settlement level (beyond just a handful of early colonial settlements). Instead, I use the growth rate in the population of farmers since the decade of initial founding as a proxy for land growth over the same time period.

\subsection{Data}

The data for farmer population comes from a sample of colonial New England probate records from 1620 to 1675 (Main, Main \& Lindert, 2013). The universe covers all deceased individuals - including landowners and landless tenants - in southern New England over this time period. The probate data also includes categories for each individual's occupation, value of real property, wealth \footnote{Wealth measurements include real property as well as other types of capital.}, debt, age, and sample weight by age group \footnote{The weight equals the inverse of the probability of selection for a deceased individual of a certain age group reported in the sample. For example, a deceased individual is more likely to be older than younger. The probability of a sample probate listing an older person is higher than a younger person.}.

% \footnote{The universe excludes what are today New Hampshire and Maine.}
%\footnote{Various measurements for the value of real property are reported in sterling, real sterling, and dollar equivalents. I used the value for real sterling in this analysis and was able to crosscheck those values using the nominal figures and deflator they provided.}

%One point of concern between the stated variable of interest (i.e. land) and the data available in the sample (i.e. farmer population and value of real estate) is a discrepancy in the data generating process for each variable. 

The area of a township cadaster is a {\em stock} measurement of land; the number of deceased individuals reported in the probate is a {\em flow} measurement of the colonial population. %\todo{I need to investigate demography papers that address measurements of pop growth given limited data. I have (naturally lagged) probate data.}. 
To account for this discrepancy, I aggregated the farmer population\footnote{Each record represents a deceased individual. The farmer population of a given township equals the sum of the sample weights across all individuals listed as a farmer, artisan-farmer, or laborer. See: Occupational codes listed in the probate codebook.} by decade.
%The assumptions here are twofold:  1) an aggregate flow of deceased individuals over the course of a decade is a close approximation to the stock of farmers or land held by a township at a specific time period and 2) a decade-over-decade growth rate is a close approximation to a change in the stock of land.
Annual growth rates had many gaps, were too variable, and did not reasonably represent changes in colonial territory\footnote{Aggregating probate records by decade also suffers from fewer observations and less variation.}. The model assigns one decennial growth rate to each township. The decennial growth rates were computed using a straight-line approach\footnote{
$$
Decennial \ Growth=
\frac{ 1670 \ to \ 1676 \ Pop - First \ Settlement \ Decade \ Pop }{First \ Settlement \ Decade \ Pop} *
\frac{100}{Number \ of \ Decades}
$$}.

I also consider the measurement of farmer population to be more economically representative of land growth relative to real estate value. Seventeenth century frontier farming in the American Colonies can be characterized as a factor minimizing production function between land (capital) and labor: $F(K,N)=min[K,N]$. This type of production function treats land (K) and labor (N) as complements. It assumes that a given acre of land did not yield more output (or be anymore productive) after substituting production towards more labor. Given that assumption, any increase in agricultural production would need to be met with a one-to-one increase in {\em both} factors. 
%\todo{Descriptive Stats of Probate Data: Number of townships, aggregate population count by year and decade}

Figure B1 in Appendix B highlights in red or yellow each of the townships that were damaged or destroyed during King Philip's War. The source for township damage comes from a collection of accounts from the war. Figure B2 includes other areas of conflict in colonial New England 30 years before and after King Philip's War. For this analysis, I assigned a binary variable of 1 to any township that was either damaged or destroyed in the 17th century. Although Figure B1 distinguishes between destroyed and damaged towns, the relative magnitude of destruction is unknown. I consider any sign of war-related property damage as sufficient for indicating conflict. Figure B4 is a magnified map showing tribal settlements in yellow and ``Indian praying towns'' (Christian missions) in black. 

\subsection{Results}

Table 1 shows the estimates and standard errors for models (1) through (4). The dependent variable of interest is a binary indicator of a colonial township being damaged or destroyed. A $LOGIT$ model accounts for this binary relationship and bounds the predicted values to between zero and one. All of the models presented in Table 1 are a logistic regression measuring the relationship between the population growth rate with the likelihood of the township being damaged or destroyed. Rewriting the $p_{conflict}$ equation in Section 3 with the parameters from column (1) of Table 1, the relationship between settlement growth and the likelihood of conflict in terms of probability can be expressed as:
\begin{equation}
P{conflict}=\frac{1}{1+e^{2.49-0.065*GrowthRate_{i}}}
\end{equation}

\begin{table}[!htbp] \centering 
  \caption{Results} 
  \label{} 

\begin{tabular}{@{\extracolsep{5pt}}lD{.}{.}{-3} D{.}{.}{-3} D{.}{.}{-3} D{.}{.}{-3} } 
\\[-1.8ex]\hline 
\hline \\[-1.8ex] 
 %& \multicolumn{4}{c}{\textit{Dependent variable:}} \\ 
%\cline{2-5} 
\\[-1.8ex] & \multicolumn{4}{c}{Settlement Attacked} \\ 
\\[-1.8ex] & \multicolumn{1}{c}{(1)} & \multicolumn{1}{c}{(2)} & \multicolumn{1}{c}{(3)} & \multicolumn{1}{c}{(4)}\\ 
\hline \\[-1.8ex] 
 Settlement Growth & 0.065^{*} & 0.073^{*} & 0.060 & 0.139^{***} \\ 
  & (0.037) & (0.040) & (0.052) & (0.030) \\ 
  & & & & \\ 
 Initial Settlement Pop &  & 0.014 & -0.004 &  \\ 
  &  & (0.026) & (0.033) &  \\ 
  & & & & \\ 
 Within 20MI of &  &  & 3.615^{***} &  \\ 
 Tribal Settlement &  &  & (1.096) &  \\ 
  & & & & \\ 
 Constant & -2.486^{***} & -2.721^{***} & -3.616^{***} & -2.406^{***} \\ 
  & (0.559) & (0.729) & (1.074) & (0.156) \\ 
  & & & & \\ 
\hline \\[-1.8ex] 
Observations & \multicolumn{1}{c}{56} & \multicolumn{1}{c}{56} & \multicolumn{1}{c}{56} & \multicolumn{1}{c}{56} \\ 
%Log Likelihood & \multicolumn{1}{c}{-19.666} & \multicolumn{1}{c}{-19.527} & \multicolumn{1}{c}{-12.803} & \multicolumn{1}{c}{-224.606} \\ 
%Akaike Inf. Crit. & \multicolumn{1}{c}{43.332} & \multicolumn{1}{c}{45.055} & \multicolumn{1}{c}{33.606} & \multicolumn{1}{c}{453.211} \\ 
\hline 
\hline \\[-1.8ex] 
\textit{}  & \multicolumn{4}{r}{$^{*}$p$<$0.1; $^{**}$p$<$0.05; $^{***}$p$<$0.01} \\ 
\end{tabular} 

\end{table} 

According to baseline model (1), a settlement that doubled in size (or a 100\% growth rate) over the period from its initial settlement date to 1670-1676 was 8.1\%\footnote{$\frac{1}{1+e^{2.49-0.065}}$} more likely to be damaged or destroyed\footnote{Townships with higher growth rates \textit{decrease} the denominator which \textit{increases} the probability of a town being damaged or destroyed.}. %\todo{Can I measure severity of conflict that weighs destroyed settlements more than damaged settlements?} 
Model (2) includes the initial settlement size as a separate explanatory variable. Initial settlement size has a slight positive relationship with the likelihood conflict but is not statistically significant. Settlement growth remains positively correlated with the probability of conflict and has a similar magnitude as the baseline model. Model (3) adds a binary variable indicating whether a rival tribal settlement is within 20 miles of an English settlement. While proximity to nearest tribal settlement has a strong relationship with the likelihood of conflict, the coefficient is completely cancelled out by the model's constant, making the odds of conflict close to 50\%\footnote{$\frac{1}{1+e^{-3.616+3.615}}$}. The coefficient for settlement growth is also similar to the previous two baseline models but is not statistically significant.

Model (4) weights settlement growth by initial settlement size at the decade of its founding. The constant parameter is similar to models (1) and (2). Given a settlement's initial size at founding, settlements that doubled in size were roughly 9\% more likely to be damaged or destroyed by rival tribes. Models (2) and (4) suggest that settlements with a large presence at founding {\em and} which subsequently grew more quickly were more likely to share contestable land with rival tribes and lead to conflict. 

A 100\% increase (doubling) in township population was associated with an 8-9\% increase in the likelihood of conflict. Given township growth rates ranged from a doubling in size to a factor of 45 over the 50-year period, this relationship between growth and conflict was significant. Most townships in the probate sample grew by a factor less than 20. Figure B3 breaks out the attacked and spared settlements by growth rate. The size of the points indicate the initial settlement size at the time of founding. Townships within 20 miles of a tribal settlement are in green; townships farther than 20 miles of a tribal settlement are in red.

\section{Plymouth Settlement Analysis}\label{sec5}

The cross-township analysis suffers from an absence of data on the growth of land in English settlements measured in surface area. In order to corroborate the magnitude of growth rates observed in the cross-township analysis based on the population of farmers who entered probate, I also looked at archaeological data on livestock and pasture size over the same time period for Plymouth township to compare how growth in the number and type of livestock reconciles with the proxy variables used in the cross-township analysis. Using United States Department of Agriculture (USDA) guidance on pasture size capacity, I imputed the total number of pasture acres required to accommodate the number of livestock reported in the archaeological site.

\subsection{Plymouth Archaeological Rediscovery Project}

Craig Chartier and other members of the Plymouth Archaeological Rediscovery Project (PARP) used faunal analysis to catalogue and substantiate the number of cattle remains recorded across three sites within Plymouth township. The faunal results reported in Figure C6 show the total stock of cattle for each decade. Ratios of other types of livestock relative to each cow were also confirmed in the archaeological analysis. I applied those ratios to the total stock of cattle to derive the total number of livestock for each decade. The total number of livestock and decennial growth rate in the number of livestock are reported in the last two columns of the lower table in Figure C6.

\subsection{Pasture Size}

Using the USDA formula, I then imputed the pasture size required to accommodate the livestock from Figure C6. The USDA formula is adapted for small-scale farm use and mixed operations including foraging and crop rotation (NRCS, USDA). These assumptions are more relevant for 17th century agriculture and colonial English farming methods. I scaled down the average size of each animal\footnote{This assumes scrawnier livestock in the 17th century.} as well as reduced the utilization rate and average yield per acre to reflect a level of farm productivity that is closer to subsistence. Settlement growth, in this analysis, is entirely driven by the growth in cattle reported from archaeological evidence and the ratios of livestock weight.

The total number of acres required to house each type of livestock are reported by decade in Figure C7. The assumptions I made for average livestock weight, utilization rate, grazing days per year, and average yield per acre are listed on the top left. The pasture size imputations reported in Figure C7 are constructed using the USDA formula and total livestock figures reported in Figure C6. The decennial growth rate in pasture size for Plymouth township is reported in column 7 of Figure C7. The highlighted records in Figure C7 compare the decennial growth rates of the faunal analysis to the growth in farmer population (and real estate value) reported for Plymouth. The growth in farmer population comes from probate records used in the cross-township analysis. The growth rates show a similar pattern:  a rapid expansion in 1630-1640, moderate growth in 1650-1660, followed by a decline resulting from King Philip's War in the later part of the 1670's.

\section{Conclusion}\label{sec13}

Many factors contributed to Algonquian-English conflict from the Pequot War in the 1630's to the onset of King Philip's War in the 1670's:  colonial expansion, disparate treatment of property, the proliferation of guns, inter-colonial and inter-tribal rivalries, and the failure of diplomacy. Threats, such as weapons confiscation and perceived loss of land, drove the Wampanoag and other Algonquian tribes to violence. Peaceful resolutions to land disputes became more strained - in the absence of shared property-enforcement rules - as the area of contested land between Algonquian tribes and English settlers expanded. %Perhaps the New England frontier was a {\em terra nullius} or Hobbesian ``state of nature". Those arguments, however, were used to justify the colonization and expansion of English settlement, not propose a common approach to property with Native Americans.

Colonial growth in New England was rapid and geographically heterogenous in the early to mid 17th century. From 1620 to 1680, over 500 English townships were founded, and townships grew from a doubling in size to as much as a factor of 45. An English settlement's proximity to a nearby Algonquian settlement made conflict 50\% more likely. After accounting for proximity to the nearest Algonquian tribal settlement, English townships that grew more rapidly during the early stages of colonization were more likely to be attacked, raided, damaged, or destroyed.
%Larger initial settlements were more likely to induce conflict with relatively lower subsequent growth rates.

While initial colonial settlement size and proximity to rival tribal settlements contributed to the expansion of contested land, the strongest predictor of early colonial conflict was English settlement growth. Settlements that doubled in size were 8.1\% more likely to be attacked. Settings with weak property-rights institutions, like 17th century colonial New England, made the expansion of contested land lead to violent conflict.

\newpage
\backmatter

%\bmhead{Supplementary information}

%If your article has accompanying supplementary file/s please state so here. 

%Authors reporting data from electrophoretic gels and blots should supply the full unprocessed scans for key as part of their Supplementary information. This may be requested by the editorial team/s if it is missing.

%Please refer to Journal-level guidance for any specific requirements.

\bmhead{Acknowledgments}

This paper would not have been completed without the painstaking help and feedback from Vincent Geloso. I also received comments from my other dissertation advisors, Noel Johnson (chair) and Mark Koyama. This paper also benefited from discussions with fellow graduate students at the Public Choice Center at GMU. 

\section*{Declarations}

%Some journals require declarations to be submitted in a standardised format. Please check the Instructions for Authors of the journal to which you are submitting to see if you need to complete this section. If yes, your manuscript must contain the following sections under the heading `Declarations':

\begin{itemize}
\item No Funding was requested or received for this paper
\item The author expresses no conflict of interest related to this paper
\item Ethics approval: N/A
\item Consent to participate: N/A
\item Consent for publication: N/A
\item Data and materials referenced in this paper will be made available upon request
\item SAS, R, and Excel workbooks used to prepare the analysis are also available upon request
%\item Authors' contributions
\end{itemize}

%\noindent
%If any of the sections are not relevant to your manuscript, please include the heading and write `Not applicable' for that section. 

%%===================================================%%
%% For presentation purpose, we have included        %%
%% \bigskip command. please ignore this.             %%
%%===================================================%%
%\bigskip
%\begin{flushleft}%
%Editorial Policies for:

%\bigskip\noindent
%Springer journals and proceedings: \url{https://www.springer.com/gp/editorial-policies}

%\bigskip\noindent
%Nature Portfolio journals: \url{https://www.nature.com/nature-research/editorial-policies}

%\bigskip\noindent
%\textit{Scientific Reports}: \url{https://www.nature.com/srep/journal-policies/editorial-policies}

%\bigskip\noindent
%BMC journals: \url{https://www.biomedcentral.com/getpublished/editorial-policies}
%\end{flushleft}

\begin{appendices}

\begin{landscape}

\section{Historical Accounts}\label{secA1}
\subsection{English \& Algonquian Interactions Prior to King Philip's War}
A summary of secondary sources categorizing the types of interactions among Algonquian tribes and English settlers. 

\bigskip

\resizebox{1.7\textwidth}{!}{%
\begin{tabular}{|c|c|c|c|c|c|}
\hline
\textbf{Type of Interaction} & \textbf{Time Period} & \textbf{Algonquian People\textbackslash Tribes} & \textbf{English People\textbackslash Settlements} & \textbf{Summary of Findings} & Source  \\

\hline
\rowcolor{gray}
Trade & 1500s-1600s & Narragansett & European Traders & \multicolumn{1}{m{9cm}}{Trading, fishing, exploring. Narragansett prized European manufactured goods. Europeans demanded furs, wampum - shifting Narragansett production from commercial hunting to crafting. Beaver population depleted. Rise of wealthy Narragansett wealthy "middlemen".} & Schultz, 1999 \\

Land Use & 1627 & Wampanoag & \multicolumn{1}{m{6cm}}{Gov. William Bradford, Plymouth Colony} & \multicolumn{1}{m{9cm}}{English colonists desired additional farmland farther away from close-knit Plymouth settlement. A second grand of land was made to every resident of Plymouth to satisfy their desire for more land.} & Schultz, 1999 \\
\rowcolor{gray}
Politics \& Diplomacy & 1622 & Massasoit, Pokanoket & Plymouth Colony & \multicolumn{1}{m{9cm}}{Negotiated peace treaty (era of peace) guaranteeing English colonists' security. New ally for Wampanoag contra Narragansett.} & Schultz, 1999 \\

Land Use & 1651 & Pocasset (Nonaquaket) & \multicolumn{1}{m{6cm}}{Richard Morris, RI} & \multicolumn{1}{m{9cm}}{English colonists graze cattle on salt marsh grasses confined by water on both sides in a peninsula harvested by the Pocasset. Early test of diplomatic rhetoric, writing, and English legal discourse.} & Brooks, 2018 \\

\rowcolor{gray}
Land Use \& Trade & 1651 &  Weetamoo, Squa-Sachem of Pocasset & Portsmouth (Plymouth) & \multicolumn{1}{m{9cm}}{Portsmouth settlers relied on planting in fields. Weetamoo (Namumpum) held the role of ``cultivator of diplomacy'' working with other tribes and English settlers teaching cultivation methods.} & Brooks, 2018 \\

\hline
Land Use; Legal Chicanery & 1651 & Wamsutta \& Weetamoo of Pocasset & \multicolumn{1}{m{6cm}}{Plymouth} & \multicolumn{1}{m{9cm}}{English men "were somewhat uncomfortable in dealing with women in land transactions". English settlers strongly enforced {\em couverture}, the legal principle that all of a woman's property is transferred to her husband upon marriage, to limit the number of negotiating parties.} & Brooks, 2018 \\

\rowcolor{gray}
Politics \& Diplomacy & 1662 & Pokanoket & Colony of Rhode Island (RI) & \multicolumn{1}{m{9cm}}{Death of Ousamequin, Massasoit "great sachem" of Pokanoket. End of peaceful English-Indian relations in New England} & Schultz, 1999
\end{tabular}
}

\newpage
\thispagestyle{empty}

\resizebox{1.7\textwidth}{!}{%
\begin{tabular}{|c|c|c|c|c|c|}
\hline
\textbf{Type of Interaction} & \textbf{Time Period} & \textbf{Algonquian People\textbackslash Tribes} & \textbf{English People\textbackslash Settlements} & \textbf{Summary of Findings} & Source  \\

\hline
\rowcolor{gray}
\multicolumn{1}{m{5cm}}{Politics, Diplomacy \& Trade}  & 1675 & Narragansett & Roger Williams (RI) & \multicolumn{1}{m{9cm}}{Establish commercial and military relations with Narragansett (involved in sporadic conflict with Wampanoag)} & Schultz, 1999 \\


Disease  & 1600-1675 & New England Native Tribes & European colonists & \multicolumn{1}{m{9cm}}{Southern New England's native population declined from 90,000 in 1600 to 10,750 in 1675. Masachusett tribe warriors declined from 3,000 to 300.} & Schultz, 1999 \\

\rowcolor{gray}
Legal Chicanery  & 1676 & New England Native Tribes &  Francis Jennings, Plymouth & \multicolumn{1}{m{9cm}}{Attempts to secure land from New England natives in a legal manner. Fraudulent methods: Impose absurd amount of fines to forfeit lands in lieu of payment; allow livestock to ruin native crops; threats of violence; induce drunkenness so a native would sign a deed he was unable to understand.} & Schultz, 1999 \\

Trade \& Demographics & 1670 & Wampanoag & New England settlements & \multicolumn{1}{m{9cm}}{Children and grand-children of the English ``Great Migration'' came of age and started looking for their own land to establish farms. Increased demand for New England beef and pork to feed enslaved people in burgeoning sugar plantations of the Caribbean. Colonist population reached 60,000 (about double the total number of Natives).} & Silverman, 2019 \\

\rowcolor{gray}
Trade \& Land Use & 1650s & Narragansett & \multicolumn{1}{m{6cm}}{Dutch \& Colonies of MA, CT, RI} & \multicolumn{1}{m{9cm}}{Relative price of land shot up relative to wampum after the Dutch ``flooded the with low-quality beads. Narragansett's most valuable asset was land.'' Colony of RI made an agreement to share land use to preempt large purchases by MA investors, particularly the Atherton Company ``with malign intentions.''} & Warren, 2018 \\

Politics \& Diplomacy & 1660-1662 & Narragansett & New England Colonies & \multicolumn{1}{m{9cm}}{The restoration of the English monarchy in 1660 forced all New England colonies to renegotiate the terms and status of their settlements. RI was the first colony to recognize the new monarchy with the hopes of receiving sign-off on their charter. CT received an extremely favorable charter that included nearly all of Narragansett land agitating the neighboring colony of RI.} & Warren, 2018 \\

\rowcolor{gray}
Conflict & June 8, 1675 & Wampanoag & Plymouth & \multicolumn{1}{m{9cm}}{Three of King Philip's men were executed by Plymouth colony for the murder of the prominent Christian Indian, John Sassamon.} & Warren, 2018

\end{tabular}
}

\end{landscape}


\section{Figures}\label{secA2}

\begin{figure}[h]
%Figure here.
\caption{Towns Damaged or Destroyed}
  \includegraphics[scale=.45]{"KPW Map".jpg}  \\
%\begin{figurenotes}
%Figure notes without optional leadin.
%\end{figurenotes}
%\begin{figurenotes}[Source]
Source: American History Online
%\end{figurenotes}
\end{figure}



\begin{figure}[h]
%Figure here.
\caption{Washburn Areas of Conflict}
 \centering{\includegraphics[scale=.5]{"washburnmap".png} } \\
%\begin{figurenotes}
%Figure notes without optional leadin.
%\end{figurenotes}
%\begin{figurenotes}[Source]
Source: Washburn, Vol. 4
%\end{figurenotes}
\end{figure}

\begin{figure}[h]
%Figure here.
\caption{Logistic Regression Model}
  \includegraphics[scale=.6]{"Farmer Growth Rate Init Pop".png} 
%\begin{figurenotes}
Corresponds to Model (5) in Table 1. Growth rates are weighted by initial population.
%\end{figurenotes}
%\begin{figurenotes}[Source]
%\end{figurenotes}
\end{figure}

\begin{figure}[h]
%Figure here.
\caption{Hassanamesit and mission communities, with Missitekw, Cambridge, Boston}
  \includegraphics[scale=.4]{"Praying-towns-MAP-PR2".jpeg} \\
%\begin{figurenotes}
%Figure notes without optional leadin.
%\end{figurenotes}
%\begin{figurenotes}[Source]
Source: Brooks, Lisa. {\em Our Beloved Kin.} 2018.
%\end{figurenotes}
\end{figure}

\begin{figure}[h]
%Figure here.
\caption{Probate \& Conflict Map}
  \includegraphics[scale=.75]{"Colonial New England".png} \\

\end{figure}


\newpage

\begin{landscape}
\section{Plymouth Settlement Pasture Growth Tabulations}\label{secA3}


\begin{figure}[h]
%Figure here.
\caption{Faunal Statistics in Plymouth Colony}
  \includegraphics[scale=.75]{"cattle".png} 
%\begin{figurenotes}
%\end{figurenotes}
%\begin{figurenotes}[Source]
%\end{figurenotes}
\end{figure}

\begin{figure}[h]
%Figure here.
\caption{Pasture Growth Rates in Plymouth Colony}
  \includegraphics[scale=.8]{"acre".png} 
%\begin{figurenotes}
%\end{figurenotes}
%\begin{figurenotes}[Source]
%\end{figurenotes}
\end{figure}

\end{landscape}



%%=============================================%%
%% For submissions to Nature Portfolio Journals %%
%% please use the heading ``Extended Data''.   %%
%%=============================================%%

%%=============================================================%%
%% Sample for another appendix section			       %%
%%=============================================================%%

%% \section{Example of another appendix section}\label{secA2}%
%% Appendices may be used for helpful, supporting or essential material that would otherwise 
%% clutter, break up or be distracting to the text. Appendices can consist of sections, figures, 
%% tables and equations etc.

\end{appendices}

%%===========================================================================================%%
%% If you are submitting to one of the Nature Portfolio journals, using the eJP submission   %%
%% system, please include the references within the manuscript file itself. You may do this  %%
%% by copying the reference list from your .bbl file, paste it into the main manuscript .tex %%
%% file, and delete the associated \verb+\bibliography+ commands.                            %%
%%===========================================================================================%%

\bibliography{sn-bibliography}% common bib file
%% if required, the content of .bbl file can be included here once bbl is generated
%%\input sn-article.bbl

%% Default %%
%%\input sn-sample-bib.tex%
\begin{thebibliography}{1}

%\bibitem{acem} Acemoglu, Daron. ``Why not a political Coase theorem?  Social conflict, commitment, and politics."  {\em Journal of Comparative Economics}. Vol. 31, Issue 4. (September 2003). pp. 620-652.
%\smallskip
\bibitem{acemr} Acemoglu, Daron, and James A. Robinson. {\em The Narrow Corridor: States, Societies, and the Fate of Liberty}. Penguin Press. New York, NY. (2019).
\smallskip
%\bibitem{acemo} Acemoglu, Daron, Simon Johnson, and James A. Robinson. ``The Colonial Origins of Comparative Development:  An Empirical Investigation."  {\em American Economic Review}. Vol. 91, No. 5. (December 2001). pp. 1369-1401.
%\smallskip
\bibitem{alston} Alston, Lee J. and Bernardo Mueller. ``Property Rights, Land Conflict, and Tenancy in Brazil."  {\em National Bureau of Economic Research}. Working Paper 15771. (March 2010).
\smallskip
\bibitem{amer} American History Online. ``King Philip's War."  Accessed Online: September 14, 2016. \url{http://online.infobase.com/HRC/Search/ImageDetails/2?imageId=58304}.
\smallskip
\bibitem{statren} Anderson, Terry L.. ``Economic Growth in Colonial New England: `Statistical Renaissance'."  {\em The Journal of Economic History}. Vol. 39. No. 1. The Tasks of Economic History. (March 1979). pp. 243-357.
\smallskip
%\url{https://drive.google.com/file/d/0B24OUgEFZDCQZFRmamVWaE13LVk/view?usp=sharing}

\bibitem{anderson} Anderson, Virginia DeJohn. {\em King Philip's Herds: Indians, Colonists, and the Problem of the Livestock in Early New England.} The William and Mary Quarterly. Third Series. Vol. 51. No. 4. (October 1994). pp. 601-624.
\smallskip
%\url{https://drive.google.com/file/d/0B24OUgEFZDCQY1RVeEc3c0hKblE/view?usp=sharing}
\bibitem{baker} Baker, Matthew J.. ``An Equilibrium Conflict Model of Land Tenure in Hunter-Gatherer Societies." {\em Journal of Political Economy}. Vol. 111, No. 1. (2003). pp. 124-171.
\smallskip
\bibitem{brooks} Brooks, Lisa. {\em Our Beloved Kin: A New History of King Philip's War.}  New Haven, CT. Yale University Press. (2018).
\smallskip
\bibitem{cain} Cain, Philip J., Anthony G. Hopkins. {\em British Imperialism:  Innovation and Expansion 1688-1914.}  Longman, New York. (1993).
\smallskip
\bibitem{candela} Candela, Rosolino A., Vincent J. Geloso. ``Trade or raid: Acadian settlers and native Americans before 1755.'' {\em Public Choice}. (2020). p. 1-27.
\smallskip
\bibitem{carlos1} Carlos, Ann M., and Frank D. Lewis. “Indians, the Beaver, and the Bay: The Economics of Depletion in the Lands of the Hudson’s Bay Company, 1700-1763.” The Journal of Economic History.
\smallskip
\bibitem{carlos2} Carlos, Ann M., and Frank D. Lewis. “Property Rights, Competition, and Depletion in the Eighteenth-Century Canadian Fur Trade: The Role of the European Market.” The Canadian Journal of Economics / Revue Canadienne d’Economique.
\smallskip
\bibitem{carlos3} Carlos, Ann M., and Frank D. Lewis. “Trade, Consumption, and the Native Economy: Lessons from York Factory, Hudson Bay.” The Journal of Economic History.
\smallskip
\bibitem{chartier} Chartier, Craig S.. {\em Plymouth Colony Livestock.}  Plymouth Archaeological Rediscovery Project.
\bibitem{cray} Cray, Robert E.. {\em ``Weltering in Their Own Blood": Puritan Casualties in King Philip's War.} Historical Journal of Massachusetts. Vol. 37. (2). (Fall 2009).
\smallskip
\bibitem{conning} Conning, Jonathan H. and James A. Robinson. ``Property Rights and the Political Organization of Agriculture."  {\em Journal of Development Economics.}  (2007). 82:  416-447.
\bibitem{deetz} Deetz, Patricia Scott and Christopher Fennell. {\em Plymouth Colony Archive Project.}  Historical Archaeology and Public Engagement. Department of Anthropology, University of Illinois at Urbana-Champaign. \url{http://www.histarch.illinois.edu/plymouth/index.html}.
\smallskip
\bibitem{dinec} Dincecco, Mark and Gabriel Katz. ``State Capacity and Long‐Run Economic Performance.'' {\em The Economic Journal}. Volume 126. Issue 590. February 2016. pp. 189–218.
\bibitem{hirsh} Hirshleifer, Jack. ``The Analytics of Continuing Conflict". \textit{Synthese}. Vol. 76, No. 2. {\em Formal Analysis in International Relations}. (August, 1988). pp. 201-233.
\smallskip
\bibitem{hughes} Hughes, J. R. T.. {\em Social Control in the Colonial Economy.}  Charlottesville, VA, University Press of Virginia, (1976).
\smallskip
\bibitem{johnkoy} Johnson, Noel D. and Mark Koyama. ``States and economic growth: Capacity and constraints.'' {\em Explorations in Economic History}. Volume 64. April 2017. pp. 1-20.
\smallskip
\bibitem{lepore} Lepore, Jill. {\em The Name of War:  King Philip's War and the Origins of American Identity.}  New York, NY, Alfred A. Knopf, Inc., (1998).
\smallskip
\bibitem{main} Main, Gloria, Jackson Main, and Peter H. Lindert. {\em Colonial New England Probates, 1631-1776.} ICPSR. 34940. (July 2013).
(Primary source from probates published in 17th century New England)
\smallskip
\bibitem{maier} Maier, Charles S.. {\em Once Within Borders:  Territories of Power, Wealth, and Belonging since 1500.}  The Belknap Press of Harvard University Press. Cambridge, MA. (2016).
\smallskip
\bibitem{nrcs} Natural Resources Conservation Service. USDA. {\em Balancing Your Animals with Your Forage:  Small Scale Solutions for Your Farm.}  Accessed Online:  (October 29, 2016). 
\smallskip
\bibitem{pagden} Pagden, Anthony. {\em The Burdens of Empire:  1639 to the Present.}  Cambridge University Press. New York, NY. (2015).
\smallskip
\bibitem{roback} Roback, Jennifer. ``Exchange, Sovereignty, and Indian-Anglo Relations.'' {\em Property Rights and Indian Economies: The Political Economy Forum}. ed. Terry L. Anderson. Rowman \& Littlefield Publishers, Inc. Lanham, MD. (1992).
\smallskip
\bibitem{schultz} Schultz, Eric B. and Michael J. Tougias. {\em King Philip's War: The History and Legacy of America's Forgotten Conflict.}  The Countryman Press. Woodstock, VT. (1999).
\smallskip
\bibitem{silverd} Silverman, David J.. {\em This Land Is Their Land: The Wampanoag Indians, Plymouth Colony, and the Troubled History of Thanksgiving.}  New York, NY. Bloomsbury Publishing. (2019).
\smallskip
\bibitem{silver} Silverman, David J.. {\em Thundersticks:  Firearms and the Violent Transformation of Native America.}  Cambridge, MA. The Belknap Press of Harvard University Press. (2016).
\bibitem{dem} Smith, Daniel Scott. {\em The Demographic History of Colonial New England.} The Journal of Economic History. Vol. 32. No. 1. The Tasks of Economic History. (March 1972). pp. 165-183.
\smallskip
\bibitem{steck} Steckel, Richard, Jerome Rose, Clark Spencer, and Phillip Walker. ``Skeletal Health in the Western Hemisphere from 4000 B.C. to the Present.'' Evolutionary Anthropology. 11:142–155. (2002).
\smallskip
%\bibitem{von} von Th\"{u}nen, J. H. {\em Der Isolierte Staat in Beziehung auf Landschaft und Nationalok\"{o}nomie.}  Hamburg (English translation by C. M. Wartenberg, {\em von Th\"{u}nen's Isolated State.}  Oxford:  Pergamon Press, 1966. (1826).
%\smallskip
\bibitem{warren} Warren, James A. {\em God, War, and Providence: The Epic Struggle of Roger Williams and the Narragansett Indians against the Puritans of New England.}  New York, NY. Scribner. (2018).
\smallskip
\bibitem{washburn} Washburn, Wilcomb E. {\em Handbook of North American Indians.} ``History of Indian-White Relations.'' ed. William C. Sturtevant. Vols. 4 \& 15. Washington, DC. Smithsonian Institution Scholarly Press. (1989).
\smallskip
\bibitem{winslow} Winslow, Edward. {\em Hypocrisy Unmasked.}  London, England. Rich. Cotes. Royal Exchange. (1646).

\end{thebibliography}

\end{document}